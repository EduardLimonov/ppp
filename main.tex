\documentclass[a4paper, 12pt, final]{article}
%\usepackage[UTF8]{inputenc}
\usepackage[T2A]{fontenc}
\usepackage[russian]{babel}

\usepackage{indentfirst}
\usepackage{fancyhdr}
\usepackage{graphicx}
\usepackage{amsmath, amsfonts}
\usepackage{amsmath}
\usepackage{amssymb}
\usepackage{setspace}
\usepackage{textcase} 

%\usepackage{pscyr} % Нормальные шрифты
%\usepackage[a4paper, left=12mm, right=20mm, top=32mm, bottom=32mm]{geometry}

\usepackage{color}
\usepackage{listings}
\usepackage{caption}
\DeclareCaptionFont{white}{\color{white}}
\DeclareCaptionFormat{listing}{\colorbox{gray}{\parbox{\textwidth}{#1#2#3}}}
\captionsetup[lstlisting]{format=listing,labelfont=white,textfont=white}

\usepackage{float}
\usepackage{longtable}

\parindent = 1.5em

\fancypagestyle{sty}{
	\fancyfoot{}
	\fancyhead{}
	\fancyhead[r]{\thepage} 
	\fancyhead[l]{\leftmark}
	
}

\pagestyle{sty}

\newcommand{\ppart}[1]{\newpage \section{#1} \thispagestyle{empty}}
\newcommand{\noNumPart}[1]{
	\newpage
	\section*{#1} \thispagestyle{empty} \addcontentsline{toc}{section}{#1}
	\markboth{\normalsize \MakeTextUppercase{#1}}{\thepage} 
	%\fancyfoot{}
	%\fancyhead{}
	%\fancyhead[r]{\thepage} 
	%\fancyhead[l]{ #1}
}
\newcommand\tab[1][1cm]{\hspace*{#1}}
\newcommand{\code}[3]{ \textbf{#1}:\\
	Вход: #2 \\
	Выход: #3
}

\lstset{ %
	language=Python,                 % выбор языка для подсветки 
	basicstyle=\small\sffamily, % размер и начертание шрифта для подсветки кода
	numbers=left,               % где поставить нумерацию строк (слева\справа)
	numberstyle=\tiny,           % размер шрифта для номеров строк
	stepnumber=1,                   % размер шага между двумя номерами строк
	numbersep=5pt,                % как далеко отстоят номера строк от подсвечиваемого кода
	backgroundcolor=\color{white}, % цвет фона подсветки - используем \usepackage{color}
	showspaces=false,            % показывать или нет пробелы специальными отступами
	showstringspaces=false,      % показывать или нет пробелы в строках
	showtabs=false,             % показывать или нет табуляцию в строках
	frame=single,              % рисовать рамку вокруг кода
	tabsize=2,                 % размер табуляции по умолчанию равен 2 пробелам
	captionpos=t,              % позиция заголовка вверху [t] или внизу [b] 
	breaklines=true,           % автоматически переносить строки (да\нет)
	breakatwhitespace=false, % переносить строки только если есть пробел
	escapeinside={\%*}{*)},   % если нужно добавить комментарии в коде
	extendedchars=\true,
	keepspaces=true
}
%\renewcommand{\thesection}{}

%\renewcommand{\thesubsection}{}
%\renewcommand{\thesubsubsection}{}


\begin{document}
	\thispagestyle{empty}
\begin{center}
	{
		{\Large МИНОБРНАУКИ РОССИЙСКОЙ ФЕДЕРАЦИИ\\
			\large Федеральное государственное бюджетное образовательное учреждение высшего профессионального образования\\
			\Large
			\par
			«Санкт-Петербургский политехнический университет Петра Великого»}
		\par
		Институт компьютерных наук и технологий
		\par
		Высшая школа искусственного интеллекта
	}
\end{center}
\vspace{5.2cm}
\begin{center}
	{
		\Large
		{\bf Отчет о лабораторной работе №1}
		\par
		\par 
		\large
		Дисциплина: «Генетические алгоритмы»
		
		%\onehalfspacing
		\par
		\vspace{1cm}
		\LARGE Простой генетический алгоритм
	}
\end{center}

\vspace{4cm}
{ 
	\onehalfspacing
	\centering
	\normalsize
	\parbox{2cm}
	{Сдал:\\
		Принял:}
	\parbox{3.5cm}
	{\vspace{0.4cm}
		
		\rule{3cm}{0.5pt}\\%подчеркиваем невидимую линейку
		\rule{3cm}{0.5pt}
	}
	\parbox{8.2cm}
	{ %\centering
		%\vspace{0.4cm}
		Черников С. Г. группа 3530201/80101\\
		д.т.н. проф. проф. Большаков А.А.
	}
}
{
	\vspace{0.5cm}
	\begin{center}
		Санкт-Петербург, 2021
	\end{center}
}
	%\newpage
	
	\tableofcontents
	\newpage
	
	%\input{introd}
	\input{part1}
	%	\input{zakl}
	
\end{document}